% PLEASE USE THIS FILE AS A TEMPLATE
% Check file iosart2c.tex for more examples
%
% Journal:
%   Journal of Ambient Intelligence and Smart Environments (jaise)
%   Web Intelligence and Agent Systems: An International Journal (wias)
%   Semantic Web: Interoperability, Usability, Applicability (SW)
% IOS Press
% Latex 2e

% options: jaise|wias|sw
% add. options: [seceqn,secfloat,secthm,crcready,onecolumn]


\documentclass[sw]{iosart2c}

%\documentclass[sw]{iosart2c}
%\documentclass[wias]{iosart2c}
%\documentclass[jaise]{iosart2c}

\usepackage[T1]{fontenc}
\usepackage{times}%
\usepackage{natbib}% for bibliography sorting/compressing
%\usepackage{amsmath}
%\usepackage{endnotes}
\usepackage{graphicx}
%\usepackage{tikz}
%\usetikzlibrary{snakes,arrows,shapes}

%%%%%%%%%%% Put your definitions here

% \bibliographystyle{plain} 
\bibliographystyle{unsrt} 

%%%%%%%%%%% End of definitions

\pubyear{0000}
\volume{0}
\firstpage{1}
\lastpage{1}

\begin{document}

\newcommand{\url}[1]{#1}

\begin{frontmatter}

%\pretitle{}
\title{The ChEMBL database as Linked Open Data}
\runningtitle{ChEMBL-RDF}
%\subtitle{}

%\review{}{}{}


% For one author:
\author[A]{\fnms{Egon} \snm{Willighagen}\thanks{Corresponding author.\\ E-mail: egon.willighagen@maastrichtuniversity.nl.}}
\runningauthor{E.L. Willighagen et al.}
% Two or more authors (order to be determined):
\author[A]{\fnms{Andra} \snm{Waagmeester}}
\author[B]{\fnms{Ola} \snm{Spjuth}},
\author[C]{\fnms{Peter} \snm{Ansell}},
\author[D]{\fnms{Antony} \snm{Williams}},
\author[D]{\fnms{Valery} \snm{Tkachenko}},
\author[E]{\fnms{Janna} \snm{Hastings}},
\author[F]{\fnms{John} \snm{Overington}},
\author[F]{\fnms{Anna} \snm{Gaulton}},
\author[F]{\fnms{Mark} \snm{Davies}},
\author[G]{\fnms{Bin} \snm{Chen}},
\author[G]{\fnms{David} \snm{Wild}}

\address[A]{Department of Bioinformatics - BiGCaT, Maastricht University, P.O. Box 616, UNS50 Box 19, NL-6200 MD, Maastricht, The Netherlands}
\address[B]{Department of Pharmaceutical Biosciences, Uppsala University, PO Box 591, SE-751 24, Uppsala, Sweden}
\address[C]{University of Queensland, St Lucia, Qld 4072, Australia}
\address[D]{Royal Society of Chemistry, 904 Tamaras Circle, Wake Forest, NC 27587, U.S.A.}
\address[E]{Chemoinformatics and Metabolism, European Bioinformatics Institute, POSTAL CODE, Hinxton, United Kingdom}
\address[F]{EMBL-European Bioinformatics Institute, Wellcome Trust Genome Campus, Hinxton, Cambridgeshire, CB10 1SD, United Kingdom}
\address[G]{School of Informatics and Computing, Indiana University, Bloomington, IN, U.S.A.}

\begin{abstract}
Making data available as RDF has the advantages that integration with other web resources brings:
we can link out to related data, others can link to this data, it makes the data machine readable,
is easy to extend with additional information,  and enables a representation of the data in common
vocabularies. This combination provides a clear, open interface.
This paper describes the continued conversion of data from the ChEMBL database into RDF triples.
This updated version of ChEMBL-RDF now uses recently introduced, such as the CHEMINF and CiTO ontologies,
exposes more information from the database, and is now available as dereferencable, linked data.
To demonstrate the new features, we here present a few new use cases, showing integration with
other web resources using semantic web technologies.
\end{abstract}

\begin{keyword}
% \sep 
Resource Description Framework, ChEMBL
\end{keyword}

\end{frontmatter}

%%%%%%%%%%% The article body starts:

\section{Introduction}\label{s1}

Despite the data deluge, finding new, unique, but significant patterns explaining biological
phenomena not yet understood is limited by the availability of data. Such data analysis is
beyond the scope of single data sets; it requires data integration from many sources, for
example, systems biology integrating micro-array differential expression to biological
pathways~\cite{}. Another prominent example is drug discovery where a new unique chemical entity is
searched with just the right biological properties. This too requires linking many
data sets~\cite{Samwald2011,OpenPHACTS}.

ChEMBL contains biological activities for over a million chemical entities, extracted from
literature, providing a unique resource for drug discovers~\cite{Gaulton2012,Warr2009}.
It is updated on a fairly frequent basis as the existing data is further curated and new data is added. Originally, this
data was available for download and via a web interface. The former requires setting up
the data in a relational database, while the latter limits the machine access to the data.
Two independent teams created RDF triples for the ChEMBL data, resulting in exposure
via Chem2Bio2RDF~\cite{Chen2010} and ChEMBL-RDF~\cite{Willighagen2011}.

This paper presents the history of the ChEMBL-RDF data set, details of the latest structures
and ontologies used for the RDF version of ChEMBL 13, and a few new example use cases,
exemplifying how the data set can be linked to other data sources to further support
research in the life sciences.

\section{Methods}\label{s2}

The ChEMBL version used in this paper is ChEMBL 13, which was released on 29 February 2012.
RDF triples were created from a local MySQL database to which the SQL data dump of ChEMBL was
inserted, using a PHP script, available from the source code hosting service
GitHub~\citep{ChEMBLRDFGitHub}, a process outlined earlier in a best practices note by
the international Health Care and Life Sciences XXX Working Group~\cite{Marshall2012}.

During the life time of this project, the ontologies used in ChEMBL-RDF have changed,
initially using a custom, ad-hoc ontology, but increasingly with community-proposed
ontologies, making the RDF more interoperable. The ChEMBL-RDF 13 dataset uses ontology
standards such as the Bibliography Ontology for Citation Typing Ontology for literature
references, and domain ontologies like the Protein Ontology and the Chemical Information
Ontology. Throughout this paper, various prefixes are used to simplify the RDF output and they are outlined
in Table~\ref{namespaces}.

\begin{table*}
\caption{Prefixes and their matching namespaces used in this paper.} \label{namespaces}
\begin{tabular}{ll}
\hline
\multicolumn{2}{l}{\textbf{Common Vocabularies}} \\
bibo    & Bibliography Ontology~\cite{Giasson2011} \\
        & http://purl.org/ontology/bibo/ \\
chebi   & Chemical Entities of Biological Interest~\cite{DeMatos2010} \\
        & http://purl.org/obo/owl/CHEBI\# \\
cheminf & Chemical Information Ontology~\cite{Hastings2011} \\
        & http://semanticscience.org/resource/ \\
cito    & Citation Typing Ontology~\cite{Shotton2010} \\
        & http://purl.org/spar/cito/ \\
obo / pro & OBO \& PRotein Ontology~\cite{Sidhu2006} \\
          & http://purl.obolibrary.org/obo/ \\

\multicolumn{2}{l}{\textbf{ChEMBL-RDF Namespace}} \\
chembl & http://rdf.farmbio.uu.se/chembl/onto/\# \\

\multicolumn{2}{l}{\textbf{ChEMBL-RDF Prefixes }}\\
act    & http://data.kasabi.com/dataset/chembl-rdf/activity/ \\
assay  & http://data.kasabi.com/dataset/chembl-rdf/assay/ \\
mol    & http://data.kasabi.com/dataset/chembl-rdf/molecule/ \\
res    & http://data.kasabi.com/dataset/chembl-rdf/resource/ \\
\hline
\end{tabular}
\end{table*}

To expose that ChEMBL-RDF data two approaches have been adopted. First, a SPARQL end point
hosted at Uppsala University, using the Open Source Virtuoso software. Use is free, but the
querying is capped, based on the estimated computational effort. Second, resources have
been made dereferencable using the Kasabi platform~\cite{kasabi}. This platform also provides a SPARQL
end point, but requires a user account.

An initial custom vocabulary has been defined, reflecting the concepts and links found in the
data structure of the ChEMBL database. This vocabulary has only partially been converted into
an OWL ontology, and is being phased out and replaced by common ontologies. This has not happened
for all data yet, and terms from the ChEMBL-RDF namespace (\textit{http://rdf.farmbio.uu.se/chembl/onto/\#},
prefix: \textit{chembl}) are still in use.

As a method to further test the access and possibilities of this Linked Data version of
ChEMBL, a few new uses cases have been developed for this paper.

\section{Results}\label{s3}

We here present the updated ChEMBL-RDF.

\subsection{ChEMBL-RDF}

\subsubsection{Data Statistics}

\# triples
\# links out
\# problems

The full set of triples are available from the website http://semantics.bigcat.unimaas.nl/chembl12/.

FIXME!

In contrast to earlier releases of ChEMBL-RDF, it now contains the chemical properties
provided by the ChEMBL database as calculated by ACD/Labs and XXXXX. The data includes
chemical properties like polar surface area, pKa and logP, counts for hydrogen bond donor
and acceptor, and rotational bonds. This data is provided for XXXXX structures.

\subsubsection{Data Statistics and Validation}

John/Anna: common statistics with SPARQL

\subsubsection{Added Data}

RSC's ChemSpider~\cite{Pence2010} is a free online database of over 26 million unique
chemical compounds aggregated from over 400 data sources as well as chemical data extracted
from RSC scientific articles and databases. Since its inception efforts have been made to
utilize the platform as both a deposition platform for the community to contribute data as
well as a platform for annotation and curation. Studies have shown that there are data
quality issues in many of the public compound databases~\cite{Williams2011} and ChemSpider has become a
valuable resource for curated data, especially chemical-compound name mappings. ChemSpider
is presently providing the chemical services underpinning the Open PHACTS semantic web
project~\cite{Williams2012} providing access to structure, substructure and similarity searching services
to the core architecture of the project. Specific chemical data sources containing data
mappings between ChemSpider identifiers (CSIDs) and the original data source identifiers
have been provided to the triple store, together with chemical identifiers including
validated chemical names (systematic, generic and trivial), SMILES () and InChIs (). 
The data mappings between the ChemSpider IDs and the data source IDs are released to
the community under Creative Commons licenses (CC-BY-SA 3.0; attribution should be
made to the original ChEMBL database, Open PHACTS, and ChemSpider).

\subsubsection{Data Structure}

For each of the common resource classes, a triple pattern was defined, following the
data available in the relational database. Figure~\ref{f1} shows how the various resource
classes are linked together. This section will show how data in the ChEMBL database
is exposed at an triple level.

\begin{figure}[t]
\includegraphics[width=0.45\textwidth]{figs/relations}
\caption{The various resource types found in the ChEMBL triples. Some entities are subclasses
of common classes, while others are instances.}\label{f1}
\end{figure}

The core concept in the ChEMBL database is that of the biological activity. This
is the type of information that the ChEMBL database extracts from literature.
Activities are at this moment still using an informal, custom vocabulary: the
type, fields and links to other resources are using the \textit{chembl}
namespace. While ChEMBL provides both the original activities as found in literature
and standardized values allowing comparison between studies, the triples only
make the latter available. The CiTO ontology is used to point to the paper from
which the data was extracted.

\begin{footnotesize}
\begin{verbatim}
act:a31863
 a       chembl:Activity ;
 cito:citesAsDataSource
  res:r6424 ;
 chembl:forMolecule mol:m180094 ;
 chembl:onAssay assay:a54505 ;
 chembl:relation ">" ;
 chembl:standardUnits
  "nM" ;
 chembl:standardValue
  "100000"^^xsd:float ;
 chembl:type "IC50" .
\end{verbatim}
\end{footnotesize}

More than five thousand different activity types are captured by the ChEMBL database.
The top five types are Potency (43\%), IC50 (13\%), MIC (4.6\%), Inhibition (3.7\%),
and Ki (3.6\%). The activity types in ChEMBL-RDF are currently not available as, or
using, an ontology.

The activities themselves are measured against assays, which make up a second important
resource type. Various assay types are found in the database: chembl:ADMET, chembl:Binding,
chembl:Functional, chembl:Property, and chembl:Unassigned. The assay information link
activities to a target (see below) and has a short description. The confidence information
assigned by the ChEMBL is exposed too, and was previously used to improve activity
prediction using Bayesian statistics~\cite{Willighagen2011}.

\begin{footnotesize}
\begin{verbatim}
assay:a17
 a chembl:Assay ;
 cito:citesAsDataSource
  res:r11347 ;
 chembl:hasAssayType chembl:ADMET ;
 chembl:hasConfScore "7"^^xsd:int ;
 chembl:hasDescription
  "Inhibition of ... hydroxylase" ;
 chembl:hasTarget target:t100122 .
\end{verbatim}
\end{footnotesize}

The assays measure activities against particular biological targets. The ChEMBL database
recognizes various types: pro:PR\_0000001, chembl:ADMET, chembl:CELL-LINE,
chembl:NUCLEIC-ACID, chembl:ORGANISM, chembl:SUBCELLULAR, chembl:TISSUE,
chembl:UNCHECKED, and chembl:UNKNOWN. The latter two are currently defined as
explicit types.

\begin{footnotesize}
\begin{verbatim}
target:t1
 a chembl:Target ;
 rdfs:subClassOf pro:PR_000000001 ;
 rdfs:label "Glucoamylase" , 
  "Maltase-glucoamylase, intestinal" ;
 dc:identifier "uniprot:O43451" ,
   "3.2.1.3" ;
 dc:title "Maltase-glucoamylase" ;
 chembl:classL1 "Enzyme" ;
 chembl:hasDescription
  "Maltase-glucoamylase, intestinal" ;
 chembl:hasKeyword "Glycosidase" , 
  "Membrane" , "Sulfation" ;
 chembl:organism "Homo sapiens" .
\end{verbatim}
\end{footnotesize}

For drug discovery, the drugs themselves are the main topic of study.
ChEMBL contains many different drug types, mostly small molecules,
but also peptides, proteins, antibodies, oligosaccharides, oligonucleotides, and
even cells. All compounds are represented as classes, following
the design of the CHEMINF ontology. These entities do not have a common
superclass in common, but can easily be identified as having the 
the role of being a drug. This is triplified using the OBO and ChEBI
ontologies:

\begin{footnotesize}
\begin{verbatim}
mol:m4 obo:has_role chebi:CHEBI_23888 .
\end{verbatim}
\end{footnotesize}

The entity typing is expressed
by subclassing other classes. For example, proteins subclass the
Protein concept from the PRotein Ontology (PR\_000000001), small
molecules subclass the Chemical Entity concept from the CHEMINF
ontology (CHEMINF\_000000), and oligosaccharides and oligonucleotides
subclass their respective matches in the CHEBI ontology (CHEBI\_50699
and CHEBI\_7754). For each drug compound the name and synonyms are
provided as labels. When InChI and InChIKeys are available for a drug, then these are
provided via the CHEMINF formalism (here abbreviated):

\begin{footnotesize}
\begin{verbatim}
mol:m41
 rdfs:label "CHEMBL406142" , 
   "Bis(3-[1 .... yl]-propionamide)" ;
 rdfs:subClassOf cheminf:CHEMINF_000000 ;
 cheminf:CHEMINF_000200
  m41:inchikey , m41:smiles , m41:inchi ;

m41:inchikey
 a cheminf:CHEMINF_000059 ;
 cheminf:SIO_000300
   "LMCOMIDLRGMFCZ-RIPOXUOASA-N" .

m41:smiles
 a cheminf:CHEMINF_000018 ;
 cheminf:SIO_000300
  "CC(C)C[C@@H]1N2C= .... O)C7=O)NC1=O)C2=O" .

m41:inchi
 a cheminf:CHEMINF_000113 ;
 cheminf:SIO_000300
  "InChI=1S/C82H102N .... 66+,67+,68+/m1/s1" .

chemblid:CHEMBL406142
 owl:equivalentClass mol:m41 .
\end{verbatim}
\end{footnotesize}

For small molecules, molecular properties are often available from ChEMBL, and as of ChEMBL-RDF 13
these too are exposed in triple format. Like the InChI and InChIKey, these are provided using the
CHEMINF ontology approach. Here is, for example, the ALogP value:

\begin{footnotesize}
\begin{verbatim}
mol:m1
 cheminf:CHEMINF_000200 m1:alogp .

m1:alogp
 a cheminf:CHEMINF_000305 ;
 cheminf:SIO_000300
  "3.344"^^xsd:double .
\end{verbatim}
\end{footnotesize}

For documents little information is replicated from the database, taking advantage
of PubMed Identifiers (PMIDs) and Digital Object Identifiers (DOIs). URIs for the
latter can be resolved online, providing curated information on the identified
documents. For each paper, basic properties are provided using the BIBO
ontology:

\begin{footnotesize}
\begin{verbatim}
journal:j6c706049c2e08871b7c46a6528065736
 a bibo:Journal ;
 dc:title "J. Med. Chem." .

res:r1
 a bibo:Article ;
 dc:date "2004" ;
 dc:isPartOf
  journal:j6c706049c2e08871b7c46a6528065736 ;
 bibo:issue "1" ;
 bibo:pageEnd "9" ;
 bibo:pageStart "1" ;
 bibo:pmid "14695813" ;
 bibo:volume "47" .
\end{verbatim}
\end{footnotesize}

\subsection{Linked Open Data}

To make the data a proper citizen of the semantic web, we link out to various resources,
which is visualized in Figure~\ref{2}. Triples for compounds link out to ChemSpider 
using the new complementary index and OpenMolecules RDF using InChI values, to Bio2RDF
for protein targets using the uniprot identifier, and CrossRef and Bio2RDF for literature
references using the DOIs and PubMed identifiers, respectively.

\begin{figure}[t]
\includegraphics[width=0.45\textwidth]{figs/lodgraph}
\caption{The links out of the ChEMBL-RDF data into the Linked Open Data cloud.
Edges are labeled by the predicates making the links.}\label{2}
\end{figure}

\subsubsection{Linking out to Bio2RDF}

The Bio2RDF project provides both resolvable Linked Data URIs using a generic Linked Data
server~\cite{Ansell2011} to access SPARQL endpoints for a range of scientific databases~\cite{Belleau2008}.
A number of these databases are referenced in ChEMBL, including ChEBI, Pubmed, and both the
Uniprot protein and taxonomy databases~\cite{TheUniProtConsortium2010}. These links are
vital to provide context for use cases that require a correlation between chemical
structures and other scientific data. 

For protein links out are based on the FOOBAR for the species, the EC code for proteins,
as well as UniProt identifiers.

FIXME: should we link out to the new UniProt RDF?

\begin{tiny}
\begin{verbatim}
target:t101191
 chembl:hasTaxonomy <http://bio2rdf.org/taxonomy:9606> ;
 owl:sameAs <http://bio2rdf.org/ec:2.7.11.1> ;
 owl:sameAs <http://bio2rdf.org/uniprot:Q8IYT8> .
\end{verbatim}
\end{tiny}

For papers with PubMed identifiers we also link out to Bio2RDF:

\begin{tiny}
\begin{verbatim}
res:r23 skos:exactMatch <http://bio2rdf.org/pubmed:15149661> .
\end{verbatim}
\end{tiny}

\subsubsection{Linking out to ChemSpider}

Link mappings are provided with skos:exactMatch predicates, while the ChemSpider identifiers
are also available via a CHEMINF representation:
 
\begin{tiny}
\begin{verbatim}
<http://data.kasabi.com/dataset/chembl-rdf/chemblid/CHEMBL324846>
 skos:exactMatch <http://rdf.chemspider.com/370> .
\end{verbatim}
\end{tiny}

\subsubsection{Linking out to OpenMolecules RDF}

The InChI is a unique identifier for (small) organic molecules, and has been previously used
to define unique IRIs for molecules~\cite{Bradley2009,Willighagen2011}. While IRIs are theoretically unlimited in length,
in practice web browsers and servers do limit the length of IRIs. Virtuoso is, unfortunately,
a system which supports only IRIs of up to a certain length. Therefore, only for smaller molecules
InChI-based links are created. Almost 1.3 million links are created, looking like:

\begin{tiny}
\begin{verbatim}
mol:m62687 owl:equivalentClass
 <http://rdf.openmolecules.net/?InChI=1S/CH4/h1H4> .
\end{verbatim}
\end{tiny}

Notice here the use of owl:equivalentClass to match the formalism in CHEMINF that defines molecules
are classes, rather than instances.

\subsubsection{Linking out to CrossRef}

Besides the PubMed identifiers used to link from literature references to Bio2RDF (see earlier),
ChEMBL uses DOIs alternatively, which we use to link out to the RDF provided by CrossRef~\citep{Bilder2011}:

\begin{tiny}
\begin{verbatim}
res:r2032 owl:sameAs <http://dx.doi.org/10.1016/0960-894X(96)00111-4> .
\end{verbatim}
\end{tiny}

\section{Applications}

The ChEMBL-RDF is increasingly used, and we here present a few applications. The first application
describes how it uses the SPARQL end point can be used to make the data available as linked data
via the Bio2RDF platform. The second application takes advantage of the bibliographic information
exposed as machine readable data, in calculating citation statistics. The third application shows
an integration of ChEMBL-RDF with ChemSpider to provide an extension for the decision support
platform in Bioclipse. The last application shows how Chem2Bio2RDF can combine the ChEMBL data
with other life sciences databases, showing the power of the Linked Data approach.

\subsection{Bio2RDF}

The Linked Data server used by Bio2RDF has been reconfigured for ChEMBL to provide URL
based services for standard URI resolution, along with text and link searches
\cite{WebAppGitHub}. It proxies the standard item
URIs by translating URLs between those requested by users and the URIs that are available
in the databases. For example, if the ChEMBL web application is running on the users
local machine, e.g. \url{http://localhost:8080/chembl/}, then a request for the
article with identifier ``a31863'', \url{http://localhost:8080/-chembl/article/a31863},
will be resolved from the database using the full original URI,
\url{http://data.kasabi.com/-dataset/chembl-rdf/13/activity/a31863}. If the
user requested an RDF document, using content negotiation, the original URIs will be unchanged,
however, if the user requested an HTML document, the results will contain both the
original RDF triples, represented using RDFa, with links that resolve using the users local machine.

The links services enables the ChEMBL application to derive both forward links, originating in
ChEMBL, e.g. \url{http://localhost:8080/chembl/linkstonamespace/-targetns/originalns:identifier},
and backward links, originating in other databases, such as LODD, Bio2RDF, e.g.
\url{http://bio2rdf.org/linksns/targetns/originalns:ident-ifier}, or Chem2Bio2RDF.
These services are vital to efficiently navigate the Linked Data web, as it is both
impractical and inefficient to require users to crawl the web before they can discover all of the
relevant resources. These services are currently only supplied as web services from ChEMBL and
Bio2RDF, but it is hoped that similar services will be provided by other scientific Linked Data
providers in the future. Datasets that are available in SPARQL endpoints can be queried for
links efficiently using templated queries in the ChEMBL web application.

\subsection{CitedIn}

Andra: use the SPARQL end point to find which entries in ChEMBL cite what papers
 
The link between data and formal publication is important in many areas of
attribution, scientist ranking, etc, as outline in \cite{Waagmeester2012}.
ChEMBL contains many literature references, and we wish to query this data
for CitedIn.

\subsection{Decision Support}

% Egon,Antony/Valery,Ola: develop and write up the Bioclipse Decision Support use case

Bioclipse Decision Support (Bioclipse DS)~\cite{Spjuth:2011uq} is a user-oriented tool for providing on-time and on-demand information on chemical structures, based on the Bioclipse workbench. Such information can include calculated properties, data from database queries, and results from predictive models. Bioclipse DS has previously been demonstrated on predictive modeling in drug safety assessment~\cite{Spjuth:2011uq} and also been linked to invoke and present results from distributed toxicity predictions from the OpenTox infrastructure~\cite{Willighagen:2011kx}.

In this study we extended Bioclipse DS with remote access to ChEMBL-RDF and Chemspider, which enables users to for chemical structures in Bioclipse look up near neighbors on Chemspider via the Chemspider Web API (SOAP) and query ChEMBL-RDF for properties regarding any measured interactions. The results are presented alongside predictive models in Bioclipse, and can be used as decision support when evaluating chemical structures and consider strategies for e.g. optimization.

[TODO: SCREENSHOT from Bioclipse-DS-ChEMBL-RDF]

\subsection{Chem2Bio2RDF}

David/Bin: Chem2Bio2RDF

\section{Discussion}

While we show here that the triple version of the ChEMBL data is very useful, the current triples are by no means static: ChEMBL-RDF
will keep evolve, following new developments in the Linked Open Data world. For example, it is likely that over time we will link out
to more Linked Data resources. More importantly, we wish to adopt more common ontologies, of which the BioAssay Ontology is planned
to be the first next to be adopted~\cite{Visser2011}. This should make the ChEMBL-RDF triples even more interoperable.

\section{Acknowledgements}

Tim Hodson and Zach Beauvais of Kasabi and A. L\"ovgren at the at Uppsala Biomedical Centre at Uppsala University for their
support.

\bibliography{article}

%%%%%%%%%%% The bibliography starts:
%\begin{thebibliography}{9}
%
%\bibitem{r1}
%
%\bibitem{r2}
%
%\end{thebibliography}

\end{document}
