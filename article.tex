% PLEASE USE THIS FILE AS A TEMPLATE
% Check file iosart2c.tex for more examples
%
% Journal:
%   Journal of Ambient Intelligence and Smart Environments (jaise)
%   Web Intelligence and Agent Systems: An International Journal (wias)
%   Semantic Web: Interoperability, Usability, Applicability (SW)
% IOS Press
% Latex 2e

% options: jaise|wias|sw
% add. options: [seceqn,secfloat,secthm,crcready,onecolumn]


\documentclass[sw]{iosart2c}

%\documentclass[sw]{iosart2c}
%\documentclass[wias]{iosart2c}
%\documentclass[jaise]{iosart2c}

\usepackage[T1]{fontenc}
\usepackage{times}%
\usepackage{natbib}% for bibliography sorting/compressing
%\usepackage{amsmath}
%\usepackage{endnotes}
%\usepackage{graphics}

%%%%%%%%%%% Put your definitions here

%%%%%%%%%%% End of definitions

\pubyear{0000}
\volume{0}
\firstpage{1}
\lastpage{1}

\begin{document}

\begin{frontmatter}

%\pretitle{}
\title{The ChEMBL database as Linked Open Data}
\runningtitle{ChEMBL-RDF}
%\subtitle{}

%\review{}{}{}


% For one author:
\author[A]{\fnms{Egon} \snm{Willighagen}\thanks{Huh?}}
\address[A]{Maastricht University}
\runningauthor{E.L. Willighagen et al.}

% Two or more authors:
\author[A]{\fnms{Ola} \snm{Spjuth}\thanks{}},
%\author[B]{\fnms{} \snm{}}
%\runningauthor{}
%\address[A]{}
%\address[B]{}

\begin{abstract}
This paper describes the continued conversion of ChEMBL data into RDF triples, using a few existing
ontologis, and demonstrates a few new use cases, taking advantage of its linked nature.
\end{abstract}

\begin{keyword}
% \sep 
Resource Description Framework, ChEMBL
\end{keyword}

\end{frontmatter}

%%%%%%%%%%% The article body starts:

\section{Introduction}\label{s1}

Linked Data, pharmaceutical research, Open PHACTS, ...

This data set has been previously used in~\cite{} and \cite{} and some used practices have
been described in \cite{}...

%\subsection{}\label{s1.1}

\section{Methods}\label{s2}

The ChEMBL version used in this paper is ChEMBL 12, which was released on ....

\subsection{Ontologies}

Ontologies... CHEMINF, ...

The BIBO ontology is used to represent documents.

Throughout this paper, various prefixes are used to simplify RDF output, outlined
in Table~\ref{namespaces}.

\subsection{Use cases}

John/Anna: common statistics with SPARQL

Ola: Use in BC-DS for possible adverse reactions.

Andra: CitedIn

The link between data and formal publication is important in many areas of
attribution, scientist ranking, etc, as outline in \cite{Waagmeester2012}.
ChEMBL contains many literature references, and we wish to query this data
for CitedIn.

Peter: Bio2RDF

David: Chem2Bio2RDF

\section{Results}\label{s3}

Previously, a SPARQL end point has been set up at Uppsala University.

Later, to satisfy the requirements for the Linked Open Drug Data \cite{}
network, the data was also hosted on Kasabi, a triple store hosting service
free for Open Data~\cite{}.

\# triples
\# links out
\# problems

For each of the common resource classes a triple pattern was defined, following the
data available in the relational database.

Assays .... anything

\begin{small}
\begin{verbatim}
act:a31863
      a       chembl:Activity ;
      cito:citesAsDataSource
              res:r6424 ;
      chembl:forMolecule mol:m180094 ;
      chembl:onAssay assay:a54505 ;
      chembl:relation ">" ;
      chembl:standardUnits
              "nM" ;
      chembl:standardValue
              "100000"^^xsd:float ;
      chembl:type "IC50" .
\end{verbatim}
\end{small}

Targets ... protein, other

\begin{verbatim}
\end{verbatim}

Activities ...

\begin{verbatim}
\end{verbatim}

Compounds ... Three types of compounds, peptides, small chemicals, ...

\begin{tiny}
\begin{verbatim}
<http://data.kasabi.com/dataset/chembl-rdf/molecule/m41> <http://www.w3.org/2000/01/rdf-schema#label> "Bis(3-[14-Benzyl-11-(1H-indol-3-ylmethyl)-2-isobutyl-13-methyl-5-(2-methylsulfanyl-ethyl)-3,6,9,12,15,20-hexaoxo-1,4,7,10,13,16-hexaaza-bicyclo[15.2.1]icos-18-en-8-yl]-propionamide)" .
<http://data.kasabi.com/dataset/chembl-rdf/molecule/m41> <http://www.w3.org/2000/01/rdf-schema#subClassOf> <http://semanticscience.org/resource/CHEMINF_000000> .
<http://data.kasabi.com/dataset/chembl-rdf/molecule/m41> <http://www.w3.org/2002/07/owl#equivalentClass> <http://bio2rdf.org/chebi:100041> .
<http://data.kasabi.com/dataset/chembl-rdf/chemblid/CHEMBL406142> <http://www.w3.org/2002/07/owl#equivalentClass> <http://data.kasabi.com/dataset/chembl-rdf/molecule/m41> .
<http://data.kasabi.com/dataset/chembl-rdf/molecule/m41> <http://www.w3.org/2002/07/owl#equivalentClass> <http://data.kasabi.com/dataset/chembl-rdf/chemblid/CHEMBL406142> .
<http://data.kasabi.com/dataset/chembl-rdf/molecule/m41> <http://www.w3.org/2000/01/rdf-schema#label> "CHEMBL406142" .
<http://data.kasabi.com/dataset/chembl-rdf/molecule/m41> <http://www.blueobelisk.org/chemistryblogs/smiles> "CC(C)C[C@@H]1N2C=CC(NC(=O)[C@H](Cc3ccccc3)N(C)C(=O)C(Cc4c[nH]c5ccccc45)NC(=O)C(CCC(=O)N)NC(=O)[C@@H](CCSCCSCC[C@H]6NC(=O)[C@H](CC(C)C)N7C=CC(NC(=O)[C@H](Cc8ccccc8)N(C)C(=O)C(Cc9c[nH]c%10ccccc9%10)NC(=O)C(CCC(=O)N)NC6=O)C7=O)NC1=O)C2=O" .
<http://data.kasabi.com/dataset/chembl-rdf/molecule/m41> <http://semanticscience.org/resource/CHEMINF_000200> <http://data.kasabi.com/dataset/chembl-rdf/molecule/m41/smiles> .
<http://data.kasabi.com/dataset/chembl-rdf/molecule/m41/smiles> <http://www.w3.org/1999/02/22-rdf-syntax-ns#type> <http://semanticscience.org/resource/CHEMINF_000018> .
<http://data.kasabi.com/dataset/chembl-rdf/molecule/m41/smiles> <http://semanticscience.org/resource/SIO_000300> "CC(C)C[C@@H]1N2C=CC(NC(=O)[C@H](Cc3ccccc3)N(C)C(=O)C(Cc4c[nH]c5ccccc45)NC(=O)C(CCC(=O)N)NC(=O)[C@@H](CCSCCSCC[C@H]6NC(=O)[C@H](CC(C)C)N7C=CC(NC(=O)[C@H](Cc8ccccc8)N(C)C(=O)C(Cc9c[nH]c%10ccccc9%10)NC(=O)C(CCC(=O)N)NC6=O)C7=O)NC1=O)C2=O" .
<http://data.kasabi.com/dataset/chembl-rdf/molecule/m41> <http://www.blueobelisk.org/chemistryblogs/inchi> "InChI=1S/C82H102N16O14S2/c1-47(2)39-67-77(107)89-59(73(103)87-57(25-27-69(83)99)71(101)93-63(43-51-45-85-55-23-15-13-21-53(51)55)79(109)95(5)65(41-49-17-9-7-10-18-49)75(105)91-61-29-33-97(67)81(61)111)31-35-113-37-38-114-36-32-60-74(104)88-58(26-28-70(84)100)72(102)94-64(44-52-46-86-56-24-16-14-22-54(52)56)80(110)96(6)66(42-50-19-11-8-12-20-50)76(106)92-62-30-34-98(82(62)112)68(40-48(3)4)78(108)90-60/h7-24,29-30,33-34,45-48,57-68,85-86H,25-28,31-32,35-44H2,1-6H3,(H2,83,99)(H2,84,100)(H,87,103)(H,88,104)(H,89,107)(H,90,108)(H,91,105)(H,92,106)(H,93,101)(H,94,102)/t57?,58?,59-,60-,61?,62?,63?,64?,65+,66+,67+,68+/m1/s1" .
<http://data.kasabi.com/dataset/chembl-rdf/molecule/m41> <http://semanticscience.org/resource/CHEMINF_000200> <http://data.kasabi.com/dataset/chembl-rdf/molecule/m41/inchi> .
<http://data.kasabi.com/dataset/chembl-rdf/molecule/m41/inchi> <http://www.w3.org/1999/02/22-rdf-syntax-ns#type> <http://semanticscience.org/resource/CHEMINF_000113> .
<http://data.kasabi.com/dataset/chembl-rdf/molecule/m41/inchi> <http://semanticscience.org/resource/SIO_000300> "InChI=1S/C82H102N16O14S2/c1-47(2)39-67-77(107)89-59(73(103)87-57(25-27-69(83)99)71(101)93-63(43-51-45-85-55-23-15-13-21-53(51)55)79(109)95(5)65(41-49-17-9-7-10-18-49)75(105)91-61-29-33-97(67)81(61)111)31-35-113-37-38-114-36-32-60-74(104)88-58(26-28-70(84)100)72(102)94-64(44-52-46-86-56-24-16-14-22-54(52)56)80(110)96(6)66(42-50-19-11-8-12-20-50)76(106)92-62-30-34-98(82(62)112)68(40-48(3)4)78(108)90-60/h7-24,29-30,33-34,45-48,57-68,85-86H,25-28,31-32,35-44H2,1-6H3,(H2,83,99)(H2,84,100)(H,87,103)(H,88,104)(H,89,107)(H,90,108)(H,91,105)(H,92,106)(H,93,101)(H,94,102)/t57?,58?,59-,60-,61?,62?,63?,64?,65+,66+,67+,68+/m1/s1" .
<http://data.kasabi.com/dataset/chembl-rdf/molecule/m41> <http://www.w3.org/2002/07/owl#equivalentClass> <http://rdf.openmolecules.net/?InChI=1S/C82H102N16O14S2/c1-47(2)39-67-77(107)89-59(73(103)87-57(25-27-69(83)99)71(101)93-63(43-51-45-85-55-23-15-13-21-53(51)55)79(109)95(5)65(41-49-17-9-7-10-18-49)75(105)91-61-29-33-97(67)81(61)111)31-35-113-37-38-114-36-32-60-74(104)88-58(26-28-70(84)100)72(102)94-64(44-52-46-86-56-24-16-14-22-54(52)56)80(110)96(6)66(42-50-19-11-8-12-20-50)76(106)92-62-30-34-98(82(62)112)68(40-48(3)4)78(108)90-60/h7-24,29-30,33-34,45-48,57-68,85-86H,25-28,31-32,35-44H2,1-6H3,(H2,83,99)(H2,84,100)(H,87,103)(H,88,104)(H,89,107)(H,90,108)(H,91,105)(H,92,106)(H,93,101)(H,94,102)/t57?,58?,59-,60-,61?,62?,63?,64?,65+,66+,67+,68+/m1/s1> .
<http://data.kasabi.com/dataset/chembl-rdf/molecule/m41> <http://www.blueobelisk.org/chemistryblogs/inchikey> "LMCOMIDLRGMFCZ-RIPOXUOASA-N" .
<http://data.kasabi.com/dataset/chembl-rdf/molecule/m41> <http://semanticscience.org/resource/CHEMINF_000200> <http://data.kasabi.com/dataset/chembl-rdf/molecule/m41/inchikey> .
<http://data.kasabi.com/dataset/chembl-rdf/molecule/m41/inchikey> <http://www.w3.org/1999/02/22-rdf-syntax-ns#type> <http://semanticscience.org/resource/CHEMINF_000059> .
<http://data.kasabi.com/dataset/chembl-rdf/molecule/m41/inchikey> <http://semanticscience.org/resource/SIO_000300> "LMCOMIDLRGMFCZ-RIPOXUOASA-N" .
\end{verbatim}
\end{tiny}

For documents little information is replicated from the database, taking advantage
of PubMed Identifiers (PMIDs) and Digital Object Identifiers (DOIs). URIs for the
latter can be resolved online, providing curated information on the identified
documents. For PubMed articles ...

\begin{tiny}
\begin{verbatim}
<http://data.kasabi.com/dataset/chembl-rdf/journal/j6c706049c2e08871b7c46a6528065736> <http://www.w3.org/1999/02/22-rdf-syntax-ns#type> <http://purl.org/ontology/bibo/Journal> .
<http://data.kasabi.com/dataset/chembl-rdf/journal/j6c706049c2e08871b7c46a6528065736> <http://purl.org/dc/elements/1.1/title> "J. Med. Chem." .

<http://data.kasabi.com/dataset/chembl-rdf/resource/r1> <http://www.w3.org/1999/02/22-rdf-syntax-ns#type> <http://purl.org/ontology/bibo/Article> .
<http://data.kasabi.com/dataset/chembl-rdf/resource/r1> <http://purl.org/ontology/bibo/pmid> "14695813" .
<http://data.kasabi.com/dataset/chembl-rdf/resource/r1> <http://www.w3.org/2000/01/rdf-schema#seeAlso> <http://bio2rdf.org/pubmed:14695813> .
<http://data.kasabi.com/dataset/chembl-rdf/resource/r1> <http://purl.org/dc/elements/1.1/date> "2004" .
<http://data.kasabi.com/dataset/chembl-rdf/resource/r1> <http://purl.org/ontology/bibo/volume> "47" .
<http://data.kasabi.com/dataset/chembl-rdf/resource/r1> <http://purl.org/ontology/bibo/issue> "1" .
<http://data.kasabi.com/dataset/chembl-rdf/resource/r1> <http://purl.org/ontology/bibo/pageStart> "1" .
<http://data.kasabi.com/dataset/chembl-rdf/resource/r1> <http://purl.org/ontology/bibo/pageEnd> "9" .
<http://data.kasabi.com/dataset/chembl-rdf/resource/r1> <http://purl.org/dc/elements/1.1/isPartOf> <http://data.kasabi.com/dataset/chembl-rdf/journal/j6c706049c2e08871b7c46a6528065736> .
\end{verbatim}
\end{tiny}

\# results of use cases

\section{Discussion}

outlook ...

\section{Acknowledgements}

Kasabi.

%\begin{figure}[t]
%\includegraphics{}
%\caption{Figure caption.}\label{f1}
%\end{figure}

\begin{table*}
\caption{Prefixes and their matching namespaces used in this paper.} \label{namespaces}
\begin{tabular}{ll}
\hline
Common & Vocabularies \\
cito & Citation Typing Ontology~\cite{ont:cito} \\
     & http://purl.org/spar/cito/ \\

\hline
ChEMBL-RDF & Namespaces\\
act    & http://data.kasabi.com/dataset/chembl-rdf/activity/ \\
chembl & http://rdf.farmbio.uu.se/chembl/onto/\# \\
res    & http://data.kasabi.com/dataset/chembl-rdf/resource/ \\
assay  & http://data.kasabi.com/dataset/chembl-rdf/assay/ \\
mol    & http://data.kasabi.com/dataset/chembl-rdf/molecule/ \\
\hline
\end{tabular}
\end{table*}


%%%%%%%%%%% The bibliography starts:
%\begin{thebibliography}{9}
%
%\bibitem{r1}
%
%\bibitem{r2}
%
%\end{thebibliography}

% \printbibliography

\end{document}
